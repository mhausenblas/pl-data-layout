%\documentclass[11pt,fleqn,twoside]{article}
\documentclass{llncs}

\usepackage[english]{babel}
\usepackage{alltt}
\usepackage{graphicx}
\usepackage{url}
\usepackage{subfigure}
\usepackage{listings}
\usepackage{color}
\usepackage{lscape}
\usepackage{xspace}
\usepackage{latexsym}
\usepackage{rotating} 
\usepackage{enumitem}

\usepackage{verbatim}
\usepackage{url}
\usepackage{multirow} 


\usepackage{colortbl}
\usepackage[table]{xcolor}

\makeatletter
\renewcommand{\thetable}{\thesection.\@arabic\c@table}
\@addtoreset{table}{section}
\makeatother


\definecolor{MyDarkBlue}{rgb}{0,0.08,0.45} 
\newcommand{\dv}[1] {\textcolor{blue}{[DV]\textit{#1}}}
\newcommand{\bo}[1] {\textcolor{MyDarkBlue}{[BO]\textit{#1}}}

\begin{document}

\title{Notes on Physical \& Logical Data Layouts}
	\author{
	Michael Hausenblas\inst{1} 
	}
	\institute{MapR Technologies EMEA, Ireland\\
	\email{mhausenblas@maprtech.com}
	}
\maketitle

\begin{abstract}
This article discusses principled options for physical and logical data layouts
and their implications on data processing at large scales. I should say in 
advance that these notes offer no new insights, that is, everything stated here 
has already been published elsewhere. In fact, it has been published in so many 
different places, such as blog posts, in the literature, mentioned at talks, 
spread over YouTube and Vimeo videos, uttered in hallway discussions and 
various whiteboards that its main contribution is to bring it all together in 
one place. The reader is expected to have a basic understanding in data 
management, databases and datastores as well as datashapes in general.
\end{abstract}

\section{Motivation}
\label{sec:mot}
In the following short note I shall review and discuss data layouts on a 
principled level. While no new concepts or insights are presented, the note
offers an overview and principled understanding of underlying mechanisms 
concerning data management.

\begin{figure}[h!]
\centering
\includegraphics[width=0.5\textwidth]{data-layers}
\caption{The three layers of data representation and interaction.}
\label{fig:data-layers}
\end{figure}

Conceptually, there are three levels in data management systems as shown in
Figure~\ref{fig:data-layers}:
\begin{itemize}
	\item The \emph{User Interface} level. Any database or datastore needs 
to provide a way to interact with the data under management. This can be 
something elaborate, standardised and mature as the Structured Query Language 
(SQL) found in relational database management systems (RDBMS), such as 
Oracle DB, PostgreSQL, or MySQL. This can be a RESTful interface, found in many 
NoSQL datastores, like, for example, CouchDB's API\footnote{See online 
documentation at \url{http://wiki.apache.org/couchdb/HTTP_Document_API}}. Of
course, this can also be a programming-language-level API such as the case with
Hadoop\footnote{\url{http://hadoop.apache.org/docs/current/api/org/apache/hadoop
/mapreduce/package-summary.html}}
\end{itemize}

\section{Manifestations of Data Layouts}
\label{sec:mani}




\begin{figure}[h!]
\centering
\includegraphics[width=0.9\textwidth]{taxonomy-dl}
\caption{A non-exhaustive, lightweight taxonomy for logical and physical data 
layouts and serialisation formats commonly used in the data processing 
community.}
\label{fig:taxonomy-dl}
\end{figure}

\section{Logical Layouts}
\label{sec:loglay}

\section{Physical Layouts}
\label{sec:phylay}

\section{Layering Physical and Logical Layouts}
\label{sec:laylay}


\section{Impact on Data Processing at Scale}
\label{sec:ldp}


\section{Conclusions and Challenges}
\label{sec:concl}

\section{Acknowledgements}
\label{sec:ack}
I'd like to thank Eric Brewer, whose RICON2012 keynote motivated me to write up
this short note. His keynote is available via \url{https://vimeo.com/52446728} 
and more than certainly worth it watching it.

\bibliographystyle{alpha}
\bibliography{data-proc}


\end{document}

